\capitulo{1}{Introducción}

El Centro Nacional de Investigación sobre la Evolución Humana, también
conocido como CENIEH, se ha incorporado recientemente al proyecto
europeo denominado \emph{ARIADNEplus}. 

\emph{ARIADNEplus} proporciona una infraestructura de investigación orientada a la arqueología cuyo
principal objetivo es apoyar la investigación, el aprendizaje y la
enseñanza al permitir el libre acceso a recursos y servicios digitales.
Lo consigue manteniendo un catálogo de conjuntos de datos digitales
(Figura \ref{fig:catalogAriadne}), promoviendo buenas prácticas en la gestión y
uso de datos digitales arqueológicos, y apoyando el desarrollo de nuevos
servicios innovadores para la arqueología.

\imagen{catalogAriadne}{Vista del catálogo oficial de ARIADNEplus.}{0.7}

Los conjuntos de datos almacenados en el catálogo de \emph{ARIADNEplus}
son meramente descriptivos, es decir, no contienen el dato al que hacen
referencia. Es por tanto competencia de los socios discernir quién tiene
acceso a los datos reales y quién no. Podría decirse que el catálogo
actúa como un simple escaparate que permite mostrar a los
investigadores de todo el mundo información sobre el qué, cuándo y dónde
de los datos propietarios de cada socio.

\imagen{catalogAriadne-2}{Filtros de búsqueda principales del catálogo.}{0.9}

Son estos tres conceptos, el qué, cuándo y dónde, los pilares de
información sobre los que está construido el catálogo. Hacen referencia
al tipo de dato (e.g. \emph{fieldwork}), al espacio temporal en el que
se enmarca (e.g. \emph{Neolithic}) , y al lugar geográfico donde se
ubica (e.g. \emph{Sierra de Atapuerca}).

Con la realización de este proyecto se pretende llevar a cabo el proceso
de integración de los conjuntos de datos del CENIEH en
\emph{ARIADNEplus}, de forma que estos sean visibles desde su catálogo
oficial.

Además, para aplicar los conocimientos desarrollados durante toda la
carrera, se propone una infraestructura \emph{software} que permita
gestionar estos conjuntos de datos y sirva como guía a los
investigadores del CENIEH durante el mencionado proceso de integración.


