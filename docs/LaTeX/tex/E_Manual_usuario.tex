\apendice{Documentación de usuario}

\section{Introducción}

En este apartado se pretende dotar al usuario de la información
necesaria para la correcta utilización de la aplicación. 

En primer lugar, se especificarán los requisitos \emph{software} con los que el
usuario debe contar para instalar y acceder a la aplicación. Posteriormente, se
explicará paso a paso el proceso de instalación y, finalmente, se
mostrará el manual de usuario.


\section{Requisitos}
Los requisitos de usuario varían en función del modo de instalación que
vaya a llevar a cabo el usuario. Existen tres modos de instalación:
Manual, \emph{Docker} y \emph{Kubernetes}.

Como es lógico, además de los requisitos que se van a mostrar a
continuación, se debe contar con un \textbf{navegador web} desde el que
se pueda acceder a la aplicación web.


\subsection{Manual}

Si se escoge la opción manual hay que estar seguro de que el servidor
cumple con todos y cada uno de los siguientes \textbf{requisitos}:

\begin{itemize}
\tightlist
\item
  Sistema Operativo Linux \cite{linux:web}
\item
  Apache HTTP Server \cite{apache:install} (con el módulo \emph{rewrite} activado).
\item
  MySQL / MariaDB v5.0 \cite{mysql:install} o superior.
\item
  PHP v5.4 \cite{php:install} o superior con las sisguientes extensiones instaladas:

  \begin{quote}
  \begin{itemize}
  \tightlist
  \item
    mysqli
  \item
    exif
  \item
    curl
  \item
    mbstring
  \end{itemize}
  \end{quote}
\item
  ImageMagick \cite{imagemagick:install} (Tratamiento de imágenes)
\end{itemize}

\subsection{Docker}

En este caso, solo es necesario un único \textbf{requisito}:

\begin{itemize}
\tightlist
\item
  \emph{Docker} \cite{docker:install} (Probado con la versión 19.03.6).
\end{itemize}

\subsection{Kubernetes}

Si se pretende utilizar \emph{Kubernetes} para el despliegue de la
infraestructura se requiere:

\begin{itemize}
\tightlist
\item
  \emph{Docker}  \cite{docker:install} (Probado con la versión 19.03.6).
\item
  La herramienta de línea de comandos de \emph{Kubernetes},
  \emph{kubectl} \cite{kubectl:install} (Probado en v1.18.2).
\item
  \emph{Kustomize} \cite{kustomize:repo} (probado en v3.1.0)
\end{itemize}

\section{Instalación}

Como se ha comentado en el apartado anterior, existen tres posibilidades
distintas para instalar la aplicación en un servidor: \emph{Manual},
\emph{Docker} o \emph{Kubernetes}.

\subsection{Manual}
El primer paso consiste en \textbf{configurar el servidor}. Para ello,
hay que seguir una serie de indicaciones:

\begin{enumerate}
\def\labelenumi{\arabic{enumi}.}
\tightlist
\item
  \textbf{Crear la base de datos (DB) MySQL} desde un usuario con
  permisos suficientes como para poder realizar operaciones sobre ella.

  \begin{itemize}
  \item
    Durante el proceso, conviene apuntar los siguientes datos:

    \begin{quote}
    \begin{itemize}
    \tightlist
    \item
      \emph{Hostname} donde se encuentra alojada la DB.
    \item
      Nombre de la DB.
    \item
      Nombre del usuario de la DB.
    \item
      Contraseña de usuario de la DB.
    \end{itemize}
    \end{quote}
  \item
    La base de datos ha de estar codificada en {utf8}.
\begin{verbatim}
sudo mysql -u root -
CREATE DATABASE omekadb CHARACTER SET utf8mb4 \
  COLLATE utf8mb4_unicode_ci;
CREATE USER 'usuario'@'localhost' IDENTIFIED BY 'contraseña';
GRANT ALL ON omeka.* TO 'usuario'@'localhost' \
  IDENTIFIED BY 'contraseña' WITH GRANT OPTION;
FLUSH PRIVILEGES;
EXIT;
\end{verbatim}
\end{itemize}
\item
  \textbf{Descargar} la version 2.7.1 de \textbf{Omeka}, desde su {[}web
  oficial{]}(\url{https://omeka.org/classic/download/}) o desde su
  {[}repositorio oficial{]}(\url{http://github.com/omeka/Omeka}) en
  GitHub.
\begin{verbatim}
cd /tmp && wget \
  https://github.com/omeka/Omeka/releases/download/v2.7.1/omeka-2.7.1.zip
\end{verbatim}
\item
  \textbf{Descomprimir} el fichero \emph{.zip} recién descargado sobre un
  directorio desde donde podamos trabajar.  
\begin{verbatim}
unzip omeka-2.7.1.zip -d <directorio_de_trabajo>
\end{verbatim}
\item
  Desde el directorio escogido, buscar el fichero {db.ini} y
  \textbf{sustituir los valores 'XXXXX' por los datos de la base de
  datos} (anotados en el paso 1).
\begin{verbatim}
cd <directorio_de_trabajo>
nano db.ini
\end{verbatim}
  No es necesario modificar los parámetros `prefix` o `port`.
\begin{verbatim}
[database]
host     = "localhost"
username = "usuario"
password = "contraseña"
dbname   = "omekadb"
prefix   = "omeka_"
charset  = "utf8"
;port     = ""
\end{verbatim}
\item
  \textbf{Descargar} el contenido del
  \href{https://github.com/gcm1001/TFG-CeniehAriadne}{repositorio del
  proyecto}.
\begin{verbatim}
cd /tmp && wget \
  https://github.com/gcm1001/TFG-CeniehAriadne/archive/master.zip
\end{verbatim}
\item
  \textbf{Descomprimir} las carpetas {/omeka/plugins} y {/omeka/themes}
  del fichero {.zip} recién descargado.
\begin{verbatim}
unzip master.zip 'TFG-CeniehAriadne-master/omeka/plugins/*' \
  'TFG-CeniehAriadne-master/omeka/themes/*' \
  -d <*directorio_de_trabajo*>
\end{verbatim}
\item
  Desde el directorio de trabajo, \textbf{reemplazar las carpetas
  originales} \emph{plugins} y \emph{themes} por las previamente
  descargadas.
\begin{verbatim}
cd <*directorio_de_trabajo*>
rm -rf ./plugins ./themes
sudo cp -r ./TFG-CeniehAriadne-master/omeka/* .
rm -rf ./TFG-CeniehAriadne-master
\end{verbatim}
\item
  Mover todo el contenido del directorio de trabajo a la carpeta del
  servidor Apache.
\begin{verbatim}
mv -r <*directorio_de_trabajo*>/* <*directorio_del_servidor*>
\end{verbatim}
\item
  \textbf{Dar permisos de lectura y escritura sobre todo el contenido de
  la aplicación}.
\begin{verbatim}
cd <*directorio_del_servidor*>
sudo chown -R www-data:www-data <*directorio_de_trabajo*>
sudo chmod -R 755 <*directorio_de_trabajo*>
\end{verbatim}
\item
  \textbf{Configurar el servidor Apache}:
\begin{quote}
10.1. \textbf{Crear el archivo de configuración} correspondiente a la
aplicación.
\begin{verbatim}
nano /etc/apache2/sites-available/omeka.conf
\end{verbatim}
Cambiar los valores "\emph{DocumentRoot}" y "\emph{ServerName}". ..
code-block:
\begin{verbatim}
<VirtualHost *:80>
     ServerAdmin [email protected]
     DocumentRoot <directorio_del_servidor>
     ServerName <nombre_del_servidor>

     <Directory /var/www/html/omeka/>
          Options FollowSymlinks
          AllowOverride All
          Require all granted
     </Directory>

     ErrorLog ${APACHE_LOG_DIR}/error.log
     CustomLog ${APACHE_LOG_DIR}/access.log combined

</VirtualHost>
\end{verbatim}
\item
  \textbf{10.2 Activar el sitio y el módulo rewrite} y \textbf{reiniciar el
  servidor} para aplicar los cambios.
\begin{verbatim}
a2ensite omeka.conf
a2enmod rewrite
systemctl restart apache2.service
\end{verbatim}
\end{quote}
\end{enumerate}

Desde este instante, \textbf{la aplicación será accesible desde el
navegador} (puerto 80). El último paso consiste en \textbf{completar la
instalación guiada desde el navegador}, disponible en el directorio
{/install} (e.g. {http://aplicacion.es/install}).

Una vez instalada la aplicación, para poder disfrutar de todas las
mejoras propuestas en este proyecto, se deben instalar tanto los
\emph{plugins} como el tema propuesto (ver
\protect\hyperlink{instalar-complementos-plugins}{Instalar complementos
(plugins)} e \protect\hyperlink{instalar-temas-themes}{Instalar temas
(themes)})

Por temas de seguridad, conviene eliminar la carpeta {/install/} del
directorio raíz una vez terminada la instalación de la aplicación.

\subsection{Docker}

Para proceder al despliegue \textbf{se deben descargar}, del
\href{https://github.com/gcm1001/TFG-CeniehAriadne}{repositorio del
proyecto}, los siguientes ficheros:

\begin{itemize}
\tightlist
\item
  {/Dockerfile}
\item
  {/docker-compose.yml}
\item
  {/ConfigFiles/*}
\item
  {/omeka/plugins/*}
\end{itemize}

\textbf{Mantén los subdirectorios intactos.}

A continuación debes \textbf{compilar la imagen}. Para ello, desde el
directorio donde hayas almacenado la descarga anterior, ejecuta el
siguiente comando:

\begin{verbatim}
docker build -t nombre_imagen:tag .
\end{verbatim}

\textbf{Recuerda muy bien el nombre de la imagen y el tag que pongas}
porque será necesario para el siguiente paso, que consiste en configurar
el archivo {docker-compose.yml}.

En él, solo tenemos que cambiar la etiqueta {image} del servicio
{omeka\_app} con el nombre y el tag de la imagen recién compilada:

\begin{verbatim}
...
    omeka_app:
        image: nombre_imagen:tag
\end{verbatim} 

Si se ha publicado la imagen en \emph{DockerHub}, se puede hacer
referencia a esta indicando el nombre de usuario seguido de la imagen
(e.g. username/nombre\_de\_mi\_imagen:tag).

Elimina el servicio {omeka-db-admin} si tu servidor está destinado a
producción. Este servicio incorpora la herramienta \emph{PhpMyAdmin} a
la infraestructura, la cual tiene un alto grado de vulnerabilidades.

Por último, crear los \emph{secrets} correspondientes a las contraseñas
de la base de datos:

\begin{verbatim}
echo 'contraseña_usuario_db' | docker secret create omeka_db_password -
echo 'contraseña_root_db' | docker secret create omeka_db_root_password -
cp configFiles/db.ini.example configFiles/db.ini
\end{verbatim} 

Debes modificar el fichero recién creado {db.ini} con los datos de la
base de datos. Ten en cuenta que la contraseña que introduzcas en el
fichero tiene que coincidir con la del \emph{secret} recién creado.

Ahora ya se puede desplegar la infraestructura ejecutando el siguiente
comando desde el directorio de trabajo (donde se encuentra la descarga
del primer paso).

\begin{verbatim}
docker stack deploy -c docker-compose.yml nombre_del_entorno
\end{verbatim} 

Desde este instante la aplicación es accesible desde el navegador
(puerto 80). Los siguientes pasos son los mismos que en la
\href{Manual}{instalación manual}.

\subsection{Kubernetes}

El primer paso para desplegar la aplicación mediante \emph{Kubernetes}
es montar nuestra imagen \emph{Docker} (Sigue los primeros pasos del
punto anterior, \textbf{hasta la compilación de la imagen}).

El siguiente paso consiste en desplegar la aplicación. Para esta tarea
utilizo el gestor de objetos \emph{Kustomize}. Por ello, deberás contar
con dicha herramienta. Además debes estar en posesión de los siguientes
ficheros alojados en este repositorio:

\begin{itemize}
\tightlist
\item
  /kustomization.yaml
\item
  /patch.yaml
\item
  /gke-mysql/*
\item
  /gke-omeka/*
\item
  /configFiles/db.ini
\end{itemize}

Ahora, debes crear el {secret} que contendrá todos los datos privados
necesarios para crear la la base de datos (nombre de la base de datos,
nombre de usuario, contraseña de usuario y contraseña root).

Antes de ejecutar los siguientes comandos debes crear las
\emph{variables de entorno} que se están utilizando: DB\_PASSWORD,
DB\_ROOT\_PASSWORD, DB\_USERNAME y DB\_DATABASE.

\begin{verbatim}
kubectl create secret omeka-db \
--from-literal=user-password=$DB_PASSWORD \
--from-literal=root-password=$DB_ROOT_PASSWORD \
--from-literal=username=$DB_USERNAME \
--from-literal=database=$DB_DATABASE
\end{verbatim}

Además debemos crear el \emph{configmap} que almacenará todo el
contenido del fichero de configuración {db.ini} (no necesitas
modificarlo ya que este emplea las variables de entorno utilizadas en
los comandos anteriores).

\begin{verbatim}
kubectl create configmap db-config \
--from-file ./configFiles/db.ini
\end{verbatim}

Por último, debemos indicar el identificador de nuestra imagen
\emph{Docker} en el fichero {/gke-omeka/deployment.yaml}.

\begin{verbatim}
...
    spec:
      containers:
      - image: nombre_imagen:tag
...
\end{verbatim}

Tras esto, solo faltaría ejecutar, desde el directorio raíz, el
siguiente comando:

\begin{verbatim}
kustomize build . | kubectl apply -f -
\end{verbatim}
Desde este instante la aplicación es accesible desde el navegador
(puerto 80). Los siguientes pasos son los mismos que en la
\href{Manual}{instalación manual}.


\section{Manual del usuario}

Gracias a \emph{Read the Docs} es posible acceder al manual de usuario desde el siguiente enlace:

\begin{mdframed}[outerlinecolor=red,outerlinewidth=2pt,linecolor=mucolor,middlelinewidth=3pt,roundcorner=10pt,nobreak=true]
  \begin{center}
    \url{https://tfg-ceniehariadne.rtfd.io/es/latest/anexos/E_Manual_usuario.html#manual-de-usuario}
  \end{center}
\end{mdframed}

\imagen{manualdeusuario}{Página principal del manual de usuario.}{1}




