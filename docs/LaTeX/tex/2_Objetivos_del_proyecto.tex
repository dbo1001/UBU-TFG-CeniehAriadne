\capitulo{2}{Objetivos del proyecto}

En este apartado se indican, en primer lugar, los objetivos generales
fijados durante el comienzo del proyecto. Seguido de estos, se describen
los objetivos específicos, que se corresponden con los pasos previos que
se han tomado para alcanzar las metas previamente fijadas.

\section{Objetivos generales}\label{obj.gen}

\begin{itemize}
\tightlist
\item
  Integrar los conjuntos de datos propuestos por el CENIEH en
  \emph{ARIADNEplus}.
\item
  Proporcionar al CENIEH una infraestructura \emph{software} que
  permita:

  \begin{quote}
  \begin{itemize}
  \tightlist
  \item
    Gestionar sus (meta)datos en la integración con \emph{ARIADNEplus}.
  \item
    Transformar sus esquemas de (meta)datos a un esquema estandarizado
    compatible con \emph{ARIADNEplus}.
  \item
    Compartir los datos de forma que estos sean accesibles desde el
    exterior.
  \item
    Facilitar la integración de los (meta)datos en \emph{ARIADNEplus}.
  \end{itemize}
  \end{quote}
\end{itemize}

\section{Objetivos específicos}\label{obj.esp}

\begin{itemize}
\tightlist
\item
  Estudiar el proyecto ARIADNEplus, poniendo especial incapié en el
  proceso de integración de los datos.
\item
  Estudiar todos los conjuntos de datos involucrados en el proyecto.
\item
  Diseñar e implementar un esquema de metadatos que satisfaga las
  necesidades de ambas partes, es decir, pueda ser transformado al
  modelo objetivo (\emph{AO-CAT}) y, además, tenga la capacidad de
  representar fehacientemente los conjuntos de datos propuestos por el
  CENIEH.
\item
  Encontrar una aplicación \emph{software} que cumpla con un mínimo de
  requisitos:

  \begin{quote}
  \begin{itemize}
  \tightlist
  \item
    Sea \emph{software} libre.
  \item
    Tolere un esquema de metadatos compatible con \emph{CIDOC-CRM} o
    alguna de sus variantes utilizadas por \emph{ARIADNEplus} como \emph{ACDM}
    o \emph{AO-CAT}.
  \item
    Cuente con un sistema de importación y exportación de metadatos.
  \end{itemize}
  \end{quote}
\item
  Adaptar la aplicación seleccionada a las necesidades del proyecto a
  través del desarrollo de complementos (\emph{plugins}).
\item
  Estudio y uso de \emph{Zend Framework}.
\item
  Aplicar la arquitectura MVC
  (\emph{Model}-\emph{View}-\emph{Controller}) en el desarrollo de los
  \emph{plugins}.
\item
  Estudio y uso de lenguajes empleados para el desarrollo web como
  \emph{PHP}, \emph{HTML}, \emph{JavaScript}, \emph{jQuery} y
  \emph{CSS}.
\item
  Utilizar bases de datos relacionales \emph{MySQL} (\emph{MariaDB}).
\item
  Crear un entorno de desarrollo en *Google Cloud* haciendo uso de *Google Kubernetes Engine*.
\item
  Trabajar con \emph{Docker} para facilitar el despliegue de la
  infraestructura sobre los entornos de trabajo.
\item
  Aplicar técnicas de integración continua a través de herramientas como
  \emph{GitHub Actions} o \emph{Codacy}.
\item
  Aprender a utilizar el conjunto de herramientas alojadas en los entornos de investigación
  virtuales emph{VRES} de la infraestructura \emph{D4Science}, en concreto, \emph{Vocabulary
  Matching Tool} y \emph{X3ML Mapping Tool}.
\item
  Utilizar como sistema de documentación continua \emph{Read the Docs}.
\end{itemize}

